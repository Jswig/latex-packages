% adapted from Dana C. Ernst https://gist.github.com/dcernst/1827406

\documentclass[12pt]{article}
 
\usepackage{mathpb}


\title{CLASS- Assignment X} % replace X with the appropriate number
\author{name - email}
\setcounter{section} {0} % used to start numbering of sections at appropriate number X+1

\begin{document}

\maketitle

\section*{Example}

\begin{theorem}{1}
$\N \subset \Z \subset \Q \subset \R \subset \C \subset \Ha$
\end{theorem}

\begin{exercise}{2}
    Let $z \in \C$. Then $z = a\alpha + i\beta$ for some $\alpha, \beta \in \R$, 
    and $\Rea(z) = \alpha$, $\Ima(z) = \beta$
\end{exercise}

\begin{lemma}{3}
    Denote $\card(\N)  = \aleph_0$. Then $\card \R = 2^{\aleph_0}$
    $\mathscr{P}$
\end{lemma}

\end{document}
